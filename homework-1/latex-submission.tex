\documentclass{article}
\usepackage{graphicx}
\usepackage[margin=0.5in]{geometry}
\usepackage{listings}

\title{CAAM 519, Homework \#1: \LaTeX\ Submission}
\author{\texttt{wnl1}}
\date{September 7, 2022}

\begin{document}

\maketitle

\section{Communicating with remote machines via ssh}

\begin{verbatim}
1:
weiping@weiping-VirtualBox:~$ ssh wnl1@ssh.clear.rice.edu

2:
The authenticity of host 'ssh.clear.rice.edu (128.42.124.180)' can't be established.
ECOSA key fingerprint is SHA256:5KMG+chd+4/0+nMB4d+xGS7Yq0HepQ+QdULZUMGQyao.
Are you sure you want to continue connecting (yes/no)? yes
Warning: Permanently added "ssh.clear.rice.edu,128.42.124.180' (ECDSA) to the list of known hosts.
The Rice University Network - Unauthorized access is prohibited
Password:
Duo two-factor login for wnl1

Enter a passcode or select one of the following options:

 1. Duo Push to XXX-XXX-3210
 2. Phone call to XXX-XXX-3210
 3. SMS passcodes to XXX-XXX-3210

Passcode or option (1-3): 1
Success. Logging you in...
The Rice University Network
 ===========================
 Unauthorized use is prohibited.

 This computer system is for authorized users only. Individuals using this
 system without authority or in excess of their authority are subject to
 having all their activities on this system monitored and recorded or
 examined by any authorized person, including law enforcement, as system
 personnel deem appropriate. In the course of monitoring individuals
 improperly using the system or in the course of system maintenance, the
 activities of authorized users may also be monitored and recorded. Any
 material so recorded may be disclosed as appropriate. Anyone using this
 system consents to these terms.

 Problems and/or questions should be submitted via the problem tracking
 system form: http://helpdesk.rice.edu

CURRENT USAGE AND LOAD ON THE COMPUTE NODES:
  Mon Sep 12 18:20:01 CDT 2022

 System                     # Users         Load( 5, 10, 15 minute)
   agate.clear.rice.edu         4             0.27, 0.31, 0.27
   amber.clear.rice.edu         2             0.23, 0.37, 0.38
   cobalt.clear.rice.edu        1             0.05, 0.03, 0.05
   jade.clear.rice.edu          4             0.10, 0.10, 0.11
   onyx.clear.rice.edu          4             0.03, 0.26, 0.29
   opal.clear.rice.edu          6             0.15, 0.16, 0.15
   pyrite.clear.rice.edu        5             1.07, 1.18, 1.39

NOTE: !!!!!!!!!!!!!!!!!!!!!!!!!!!!!!!!!!!!!!!!!!!!!!!!!!!!!!!!!!!!!!!!!!

   Please log an RT ticket for any issues you may have.

   Please log a ticket at https://help.rice.edu if you have any of the following:
        Feel documentation 1s lacking.
        Have trouble getting into the system.
        Feel the system is missing a tool.

   NOTE: If you don't have a home drive when using clear, please
         create a ticket in help.rice.edu Be sure to put
         "need clear home directory" in the subject and one will
         be setup for you. Include your Course (dept and number) or
         a faculty sponsor if not for a course.

   FACULTY: Please help us pre-populate home drives for your class.
   Provide us with your course number in an RT ticket like Note above.

   NOTE: rclone added to clear. More information on kb.rice.edu soon.

   NOTE: ghe (haskell) updated on clear. Run "PATH=$PATH:/clear/apps/ghc/bin" to use.

   CLEAR NEWS -- https://kb.rice.edu/internal/page.php?id=71856
   Tips and Hints -- https://kb.rice.edu/internal/page.php?id=71857

Could not chdir to home directory /storage-home/w/wnl1: No such file or directory
/usr/bin/id: cannot find name for group ID 72496

3:
[wnl1@pyrite /]$ echo $HOSTNAME
pyrite.clear.rice.edu
\end{verbatim}

\section{A script to build a LaTeX document while hiding auxiliary files}

\begin{lstlisting}[language=bash, basicstyle=\ttfamily, caption={Bash script to build LaTeX document.}]
#!/bin/bash

if [ ! -d .build ]
then
	mkdir .build
fi

pdflatex $1.tex
mv $1.aux $1.log .build
\end{lstlisting}

The first line is a shebang command that tells the compiler to execute the script using the Bash shell. The next statement asks if the hidden directory ".build" does not exists. If it doesn't exists, then the script makes a new hidden directory called ".build". If the hidden directory ".build" does already exist, the script does nothing. Next, the script compiles a LaTeX project (whose name, denoted by \$1,  is given as an argument to the bash script) using pdflatex. This produces 3 files: the project name with the extensions ".pdf", ".aux", ".log". the files with the extensions ".aux" and ".log" are moved to the hidden directory ".build".

It is possible if someone wanted to look at the .log or .aux to help edit their LaTeX file or know additional detail about their file, it may be harder for them to access those files. Also, if someone wanted to compile a LaTeX file, but wanted the output to be something else other than a .pdf file, this script would not work. Additionally, a user could want to have additionally options (such as halt-on-error) while compiling a .tex file, which are not available using this script.

\end{document}
